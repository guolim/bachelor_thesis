\chapter{总结和展望}
\label{c:conclusion}

\section{工作总结}
图像的可逆信息隐藏算法在过去几十年中取得了飞速的发展,在可逆隐藏社区出现了诸如基
于可逆压缩的算法,基于扩展的算法,基于整数变换的算法等一批优秀的可逆隐藏算法。其
中基于预测误差扩展的算法是当今应用最广泛,也是最成功的图像可逆隐藏算法,被应用于
灰度图像的可逆隐藏中。但是现阶段的算法大多数只针对灰度图像,而在实际应用中,彩色
图像更为常见。现有的算法当然可以分别应用到彩色图像的三个通道从而实现彩色图像的可
逆隐藏,但是这样并没有充分利用彩色图像本身的通道间的相关性,所以算法仍然有可以改
进的空间。
\par
本文以2013年9月发表在Signal Processing的一篇针对彩色图像可逆隐藏的算法为基础,提
出了一种新的基于彩色图像通道相关性的可逆隐藏算法。在算法中,利用了一种特殊的通道
相关性,即三个通道的图像结构基本相同,纹理复杂度基本相似。这样在三个通道的通道内
进行一轮预测后,三个通道的相同位置的预测值应该是基本相似的。通过通道间进行一轮二
阶预测,可以得到更加准确的预测误差,从而算法的效果得到了提升。
\par
另外,在提出的算法中,使用了一种特殊的针对像素进行排序的算法。传统的像素排序算法
基本都以邻域像素的方差为准则进行排序。而在本文中,排序算法首先对当前像素的预测误
差分布函数进行了预测,然后将分布对y轴取绝对值,将分布函数的峰值同y轴的距离作为像
素预测准确程度的衡量。在文中可以看出,这种排序算法即利用了邻域像素的差值,也利用
到了它们的方差,是一种更加准确的对预测准确程度的衡量。
\par
由实验结果可以看到,文中提出的算法得到了更高的嵌入容量,更好的图像质量,算法成功
的对文献\cite{li2013reversible}进行了改进。

\section{工作展望}
利用彩色图像的通道相关性,进行通道之间的二阶预测方法虽然能得到更为陡峭的、熵更小
的直方图,但是从实验结果中可以看出,最终的改进效果并不明显,针对一副图像,不同嵌
入率下的PSNR的提升只在0.2到0.4之间,提升较小。另外,在相同嵌入率下,对超过1300副
图像的实验结果表明大部分情况下,PSNR的提升在0.25 dB到0.5 dB之间。由此可知,预测
算法对预测性能的提升在实际的隐藏算法中并没有得到充分的利用,究竟如何加以利用需要
更加深入的研究。
\par
针对提升效果微弱这一现象,我们进行了简单地分析。我们认为,反映在预测误差直方图上
的熵的提升,究竟是否如论文中猜测的那样,是在纹理复杂区域由于通道间的二阶预测而导
致的?这一问题需要更加深入细致的对预测过程进行分析,如果并不完全是这样,那么针对
像素的排序算法就需要进一步改进,以针对预测方法的改进,做出适用于该预测方法的排序
算法。
\par
随着可逆信息隐藏技术的不断发展,学者们的目光将陆续集中到针对彩色图像上来。而针对
彩色图像,如何结合彩色图像自身的特点,对其天然的通道间的相关性加以利用,将是可逆
隐藏成功的关键。本本科毕业论文提出了一种简单地利用通道相关性的方法,结合预测误差
扩展的方法,同现有方法相比在嵌入容量和图像质量上有了一定的提升。下一步的工作重点
将是针对所提算法进行进一步的分析,以充分利用预测算法带来的提升;同时如何更好的利
用彩色图像通道间的相关性也是工作的目标。
