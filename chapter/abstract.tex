
\begin{cnabstract}
信息隐藏技术是一种将数据嵌入到各种载体中的技术,在现实生活中得到了广泛的应用。而
可逆信息隐藏则是信息隐藏技术的一个特别的分支,它不仅关注要嵌入的用户数据,更关注
载体本身。它要求载体在提取出被嵌入的消息后,仍然能被完整的恢复。这就对消息的嵌入
方式提出了更高的要求。而正是由于可逆信息隐藏的这种特性,它在医学图像保护、载体篡
改认证和恢复、数字媒体版权保护中得到了广泛的应用。
\par
可逆隐藏技术在过去十几年间得到了长足的发展。学者们针对数字图像、视频、音频等载体
提出了各种各样的可逆隐藏算法。其中研究的热点集中在数字灰度图像邻域。但是在现实生
活中,使用最多的是彩色图像。现阶段针对彩色图像的可逆隐藏文献很少,针对这种情况,
本论文从彩色图像的通道相关性出发,对彩色图像的可逆隐藏技术进行了研究,提出了一种
针对彩色图像的可逆隐藏算法。
\par
算法首先结合彩色图像的通道相关性,使用了一种利用通道内预测误差进行通道间二次预测
的新预测方法,实验表明,这种预测方法能比以往算法得到更陡峭、熵更小的预测误差直方
图。随后,论文介绍了一种新的排序算法。同以往排序算法不同,该算法通过对当前像素的
预测误差概率分布进行预测,兼顾了邻域的像素的预测差值和方差,能更好的反映像素的纹
理复杂度。最后通过实验比较发现,所提出的算法在嵌入容量和图像质量上比以往算法有了
一定的提升。

\cnkeywords{可逆信息隐藏,通道相关性,图像预测}
\end{cnabstract}


\begin{abstract}
Information hiding is a widely used technology that embeds data into different 
digital carriers. Reversible information hiding, as a special branch of         
information hiding, it is not only concerned about the user's embedding data,   
but also pay attention to the carriers themselves. It requires the carriers to  
be completely recovered after extracting the embedded message. This will put    
forward higher requirements for the ways of data embedding. But because of this 
feature, reversible information hiding has been widely used in medical image    
protection, authentication and tamper carrier recovery, and digital media      
copyright protection.                                                           
\par
In the last ten years, reversible hiding has been considerably developed.      
Scholars have proposed a variety of reversible hiding algorithms for digital    
images, digital videos, audios, and other carriers. And the hot spot is         
concentrated on digital gray-scale images. But in real life, it is the color    
images that are widely used. In recent days, reversible data hiding for color   
images is a rarely studied topic. So this thesis will focus on reversible data
hiding method for color images. Based on the inter-channel correlation of color
image, a novel inter-channel prediction method is examined and a corresponding
reversible data hiding algorithm is proposed.           
\par
In the proposed method, the prediction error is obtained by an inter-channel   
secondary prediction using the prediction errors of two independent             
channels. Experiments show that this prediction method can produce shaper       
Prediction-Error-Histogram (PEH). Then, we will introduce a new pixel-sorting   
algorithm that will make predictions for the probability distribution of        
prediction error of current pixel. It can utilize both differences and the         
prediction variances of neighborhood pixels. So it will reflect the texture
complexity of current pixel better. Finally, as the experiments demonstrated,
the proposed algorithm has some improvement over the previous algorithm on both
the embedding capacity and image quality.

\keywords{Reversible Data Hiding, Inter-Channel Correlation, Image Prediction}
\end{abstract}
